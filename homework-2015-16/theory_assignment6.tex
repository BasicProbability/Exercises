\documentclass{article}

\usepackage{amsmath}
\usepackage{amssymb}
\usepackage{calc}
\usepackage{fullpage}
\usepackage{hyperref}
\hypersetup{colorlinks=true, urlcolor=blue, breaklinks=true}

%\newcommand{\prop}[1]{\lbrack \! \lbrack #1 \rbrack \! \rbrack}

\newcommand{\philip}[1]{ \textcolor{red}{\textbf{Philip:} #1}}
\newcommand{\chris}[1]{ \textcolor{blue}{\textbf{Chris:} #1}}


\title{Theory Assignment 6 -- Basic Probability, Computing and Statistics\\[2mm]
\large{Fall 2015, Master of Logic, University of Amsterdam}}

\author{}
\date{Submission deadline: Monday, October 12th, 2015, 9 a.m.}



\begin{document}
\maketitle

\paragraph{Cooperation}
Cooperation among students for both theory and programming exercises
is strongly encouraged.  However, after this discussion phase, every student writes down and submits his/her own individual solution.

\paragraph{Guidelines}
You may pick {\bf 4 exercises from exercise type I}, as well as {\bf 2 from exercise type II} for submission, i.e. you need to submit {\bf a total of 6 exercises} to be able to get all points. Numbered exercises with an exclamation mark are supposed to be a bit harder and you may challenge yourself by trying to solve them.

In the directory of your private url there is folder called `theory\_submissions'. Please upload your submission there. Your submission should be a PDF-document (use a scanner for handwritten documents!) entitled \textit{AssignmentX\_yourStudentNumber.pdf}, where \textit{X} is the number of the assignment and \textit{yourStudentNumber} is your student number. If your submission does not comply with this format, we will deduct 1 point. For each day that your submission is late, we deduct 2 points. N.B.: If multiple files are submitted for a single assignment before the deadline, the latest version will be graded.

If you have any question about the homework or if you need help, do not hesitate to contact \href{mailto:T.S.Brochhagen@uva.nl}{Thomas}.

\paragraph{Exercises}

\paragraph{Type I [4 exercises: 1.5 points per exercise]}
\begin{enumerate}
	\item Consider (i) a flip of a fair coin, (ii) a toss of a fair four-sided die, and (iii) a toss of a fair six-sided die. Let a RV $X$ encode (i), (ii) and (iii) and compute $H(X)$ in each case.
	\item A biased coin comes up heads with a probability of $\frac{2}{3}$. Compute the entropy of the outcome of six coin flips.
	\item Let $X$ encode the sum of the roll of two fair dice. Compute $H(X)$.
	\item Assume that there is an equal probability of $50\%$ for a person in a population to be male or female. Suppose further that $20\%$ of the males and $6\%$ of the females are tall (height greater than some fixed threshold). Calculate the surprisal of (i) learning that a male person is tall, (ii) a female person is tall, (iii) a tall person is female.
	\item Calculate the entropy of the following: (i) pixel values whose possible values are all integers in $[0,256]$ with uniform probability, (ii) dogs sorted by whether or not they are mammals, (iii) dogs sorted by whether they are older or not than the population's \href{https://en.wikipedia.org/wiki/Median}{median}, (iv) RV $X$ with $P(X = 0) = \frac{1}{3}, P(X = 1) = \frac{1}{4}, P(X = 2) = \frac{1}{6}, P(X = 3) = \frac{1}{6}, P(X = 4) = \frac{1}{12}$.
	\item Consider two distributions $p$ and $q$ over $\{a,b\}$. Let $p(a) = 1 - n$ and $q(a) = 1 - m$. Compute $D(p\parallel q)$ and $D(q \parallel p)$ for (i) $m = n$ and (ii) $n = \frac{1}{3}, m = \frac{1}{4}$.
\end{enumerate}

\paragraph{Type II [2 exercises: 2 points per exercise]}
\begin{enumerate}
	\item Show that relative entropy does not satisfy (i) symmetry nor (ii) triangle inequality. That is, it needs not hold that $D(X\parallel Y) \neq  D(Y \parallel X)$ nor $D(X\parallel Y) + D(Y\parallel Z) \geq D(X \parallel Z)$.
	\item Prove that the entropy of $n$ independent RVs is the sum of the entropy of the individual RVs, i.e. that  $H(X_1, ..., X_n)$ is additive if $P(\bigcap_{i = 1}^{n} X_i=x) = \Pi_{i = 1}^n P(X_i=x)$.
	\item You 5 sequences of 5 coin tosses. Thus, your data set is  [3, 5, 1, 2, 2]  where each number
	is the amount of times that heads appeared in a sequence. Assume that two coins could have generated
	each of those sequences. The first coin has parameter $ \theta = 0.6 $ and the second coin has 
	parameter $ \theta = 0.4 $. Assume that a priori both coins are equally probable to have generated 
	each sequence. Perform two iterations of EM by hand and report the posterior probability for
	each coin on the entire data set after the second iteration of EM. Clearly show the E and M steps in each iteration.
	Feel free to use a software or calculator to do the numerical computations such as computing the probability
	of the data given a parameter and all additions, multiplications and divisions.
	\item[*] Show that $D(X\parallel Y) \geq 0$.

\end{enumerate}
\end{document}
