\documentclass[a4paper,10pt,landscape,twocolumn]{scrartcl}

%% Settings
\newcommand\problemset{1}
\newif\ifcomments
\commentsfalse % hide comments
%\commentstrue % show comments

% Packages
\usepackage[english]{{../exercises}}
\usepackage{wasysym, nicefrac}

\begin{document}

\solutions

%%%%%%%%%%%%%%%%%%%%%%%%%%%%%%%%%%%%%%%%%%%%%%%%%%%

\noindent
If you think you've spotted a typo or mistake, please let us know!

\begin{exercise}[]
%	There are 11 students in a class: 4 boys and 7 girls. We need to form a group of 5 people.
	
	\begin{subex}
		That is simply $ 11 \choose 5$.
	\end{subex}

	 \begin{subex}
	 	There are $N_4 = {7 \choose 4} \times 4 = 140$ groups with exactly 4 girls and $N_5 = {7 \choose 5} = 21$ groups with exactly 5 girls. That gives $n_D = N_4 + N_5 = 161$ groups with at least 4 girls.
	 \end{subex}
	 
	 \begin{subex}
%		If you pick the group (uniformly) at random, what is the probability that there are at least 3 boys in the group?
		There are $N_3 = {3 \choose 4} \times {2 \choose 7} = 84$ groups with exactly 3 boys and $N_4 = 1 \times 7 = 7$ groups with exactly 4 boys, giving a total of $N_D = N_3 + N_4 = 91$ groups.
		The total number of 5-student groups is $N_T = {11 \choose 5} = 462$, hence the probability asked for is $p_D = \nicefrac{N_D}{N_T} \approx 0.197$.
	 \end{subex}
\end{exercise}

\begin{exercise}[Words]
	\begin{subex}
%	How many `words' of length 5 can you make using each letter of the alphabet at most once?
    There are 26 letters to use, which gives ${}_{26}P_{5} = \nicefrac{26!}{(26-5)!} = 26 \times 25 \times 24 \times 23 \times 22 = 7893600$ `words'.
	\end{subex}
	
	\begin{subex}
%	And how many if the order of the letters is irrelevant? (I.e., if we treat `words'  and `sword' as the same word.)
    Now we're looking at \emph{combinations} rather than \emph{permutations}: ${}_{26}C_{5} = {26 \choose 5} = 65780$.
	\end{subex}
	
	\begin{subex}
%	And how many words can you make if you can use every letter as many times as you like?	
    For each of the 5 positions you have 26 choices, so $26^5$.
	\end{subex}

	\begin{subex}
%	In how many unique ways can the letters in the word `error' be arranged?
    First, there are ${5 \choose 3}$ ways to position the r's. Then you can place the e in one on the $2$ remaining positions, and the o goes in the remaining spot. So ${5 \choose 3} \times 2  \times 1 = 20$ words. 
	\end{subex}

	\begin{subex}
%	 Consider a word of $n$ letters in which two letters occur more than once: $p$ and $q$ times respectively. How many unique `words' of the same length can you make of the $n$ letters?
    Let $r$ be the number of letters that occur only once (so $n = p + q + r$) and let's call the reoccuring letters $P$ and $Q$.
    We follow the argument of the previous exercise.
    First, we can position the $P$'s in ${n \choose p}$ places, after which the ${Q}$'s can be placed in ${{n-p} \choose q}$ slots. The remaining $r$ letters can be ordered in $r!$ ways, so in total there are 
    \[
      N_T = {n \choose p} \times {{n-p} \choose q} \times r!
    \]
    unique words we can make of these letters.
    
    Simplify this expression suggests a simpler argument. Expanding the binomial coefficients and writing $r = n - p - q$ gives
    \[
      N_T = \frac{n!}{p!(n-p)!} \times \frac{(n-p)!}{q!(n-p-q)!} \times (n-p-q)!
        = \frac{n!}{p!q!}
    \]
    So a simpler argument would be something like: there are $n!$ possible orderings of $n$ letters, but $p!$ of those are indistinguishable as they only shuffle $P$'s and a further $q!$ are identical because they only shuffled $Q$'s, hence $N_T = \nicefrac{n!}{p!q!}$. 
	\end{subex}
\end{exercise}

\begin{exercise}[Books]
%	 You have 3 books on complexity theory, 2 on probability theory, and 1 novel.
	\begin{subex}
%	In how many ways can the books be arranged?	
	The books are distinguishable, so there are $6! = 720$ ways to order these $6$ books.
	\end{subex}
	
	\begin{subex}
%	And what if	the books on complexity theory must be together but the other books can be arranged in any order?
    Now there are $3!=6$ ways to order the complexity-group, and $4!=24$ ways to order the complexity group and the remaining 3 books. This gives a total of $24\times 6 = 144$ orderings.
	\end{subex}
\end{exercise}

\begin{exercise}[Poker hands]
	\begin{subex}[Two-pair]
%		Two cards have one rank, two cards have another rank, and the remaining card has a third rank. Example: two 2's, two 5's and a king.
		First we choose two ranks in ${13 \choose 2}$ ways. For each of those ranks we choose 2 suits in ${4 \choose 2}^2$ ways. For the last card, we pick one of 11 ranks and one of 4 suits. In total, this gives
    	 \[  
            N_D = { 13 \choose 2} \times {4 \choose 2}^2 \times 11 \times 4 = 123 552
      	 \]
      	 ways to draw a two-pair.
      	 Since there are $N_T = {52 \choose 5} = 2598960$ possible hands, the probability of drawing a two-pair is 
      	 \[
      	   p_D = \frac{N_D}{N_T} = \frac{123552}{2598960} \approx 0.0475
    	 \]
	\end{subex}
	
	\begin{subex}[Three-of-a-kind]
%		 Three cards have one rank and the remaining two cards have two other ranks. Example: three 2's, a five and a king.
		 First we pick one of 13 ranks, and then three suits in ${3 \choose 4}$ ways. Next we pick two ranks for the remaining cards in one of ${12 \choose 2}$ ways and two suits in $4^2$ ways.
		 This gives 
		 \[
		  N_D = 13 \times {4 \choose 3} \times {12 \choose 2} \times 4^2 = 54912
		 \]
		 and as a result
		 \[
		  p_D = \frac{N_D}{N_T} = \frac{54912}{2598960} \approx 0.0211
		 \]
	\end{subex}
\end{exercise}

\vfill
\credits{Some questions are from MIT course 18-05 by Jeremy Orloff and Jonathan Bloom, see ocw.mit.edu.}
\end{document}