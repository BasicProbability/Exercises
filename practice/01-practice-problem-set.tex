\documentclass[a4paper,10pt,landscape,twocolumn]{scrartcl}

%% Settings
\newcommand\problemset{1}
\newif\ifcomments
\commentsfalse % hide comments
%\commentstrue % show comments

% Packages
\usepackage[english]{{../exercises}}
\usepackage{wasysym}

\begin{document}

\practiceproblems

{\sffamily\noindent
This week's exercises deal with sets, counting and uniform probabilities. You
do not have to hand these exercises in; they are optional and for practicing
only. If you have questions about them, please post them to the
\href{\discussionForumURL}{discussion forum} and try to help each other. We
will also keep an eye on that.
}

%%%%%%%%%%%%%%%%%%%%%%%%%%%%%%%%%%%%%%%%%%%%%%%%%%%

\begin{exercise}[]
  There are 11 students in a class: 4 boys and 7 girls. 
  We need to form a group of 5 people.
	
  % a)
  \begin{subex}
    How many different groups can you make?
  \end{subex}
  
  % b)
  \begin{subex}
  	How many different groups are possible with at least 4 girls?
  \end{subex}
  
  % c)
  \begin{subex}
    If you pick the group (uniformly) at random, 
    what is the probability that there are at least 3 boys in the group?
  \end{subex}
\end{exercise}

%---------------

\begin{exercise}[Words]

  % a)
  \begin{subex}
    How many `words' of length 5 can you make 
    using each letter of the alphabet at most once?
  \end{subex}
  
  % b)
  \begin{subex}
    And how many if the order of the letters is irrelevant? 
    (I.e., if we treat `words'  and `sword' as the same word.)
  \end{subex}
  
  % c)
  \begin{subex}
    And how many words can you make if you can use every letter as many times
    as you like?
  \end{subex}
  
  % d)
  \begin{subex}
    In how many unique ways can the letters in the word `error' be arranged?
  \end{subex}
  
  % e)
  \begin{subex}
    Consider a word of $n$ letters in which two letters occur more than once: 
    $p$ and $q$ times respectively. How many unique `words' of the same length
    can you make of the $n$ letters?
  \end{subex}
\end{exercise}

%---------------

\newpage
\begin{exercise}[Books]
  You have 3 books on complexity theory, 2 on probability theory, and 1 novel.
  
  % a)
  \begin{subex}
    In how many ways can the books be arranged?	
  \end{subex}
  
  
  % b)
  \begin{subex}
    And what if the books on complexity theory must be together but the other
    books can be arranged in any order?
  \end{subex}
\end{exercise}

%---------------

\begin{exercise}[Poker hands]
  Calculate the probability of drawing each of these poker hands.
  
  % a)
  \begin{subex}[Two-pair]
    Two cards have one rank, two cards have another rank, and the remaining
    card has a third rank. Example: two 2's, two 5's and a king.
  \end{subex}
  
  % b)
  \begin{subex}[Three-of-a-kind]
  	 Three cards have one rank and the remaining two cards have two other ranks.
    Example: three 2's, a five and a king.
  \end{subex}
\end{exercise}


\vfill
\credits{Some questions are from MIT course 18-05 by Jeremy Orloff and Jonathan
Bloom, see ocw.mit.edu.}
\end{document}