\documentclass[a4paper,10pt,landscape,twocolumn]{scrartcl}
\newcommand\problemset{3}
\newcommand\deadline{Wednesday September 19th, 21:00h}
\newif\ifcomments
\commentsfalse % hide comments
%\commentstrue % show comments
\usepackage[english]{{../exercises}}
\usepackage{wasysym,graphicx}
\begin{document}

\practiceproblems

{\sffamily\noindent
  This week's exercises deal with discrete random variables. You do not have to
  hand these exercises in; they are optional and for practicing only. If you
  have questions about them, please post them to the
  \href{\discussionForumURL}{discussion forum} and try to help each other. We
  will also keep an eye on that.
}

%%%%%%%%%%%%%%%%%%%%%%%%%%%%%%%%%%%%%%%%%%%%%%%%%%%%%%%%%%%%

\begin{exercise}[]
  Let $X$ be a random variable taking any of the values $0, 5, -5$ with
  probabilities $P(X=-5) = P(X	= 0) = 0.3$ and $P(X=5) = 0.4$. What is
  $E[X^2]$?
\end{exercise}

%---------------

\begin{exercise}[]
  Compute the pmf of the number of tails for 7 subsequent coin tosses.	
\end{exercise}

%---------------

\begin{exercise}[]
  Calculate the variance of a loaded 6-sided die that has a probability of
  $\frac 1 6$ for all odd numbered sides and $P(\{2\}) = P(\{4\}) =0.1$ and
  $P(\{6\}) = 0.3$.
\end{exercise}

%---------------

\begin{exercise}[Independence]
  Three events $A$, $B$, and $C$ are pairwise independent if each pair is
  independent. They are mutually independent if they are pairwise independent
  and in addition
  \begin{align}\label{eq:pairwise-indep}
    P(A \cap B \cap C) = P(A) P(B) P(C) \, .
  \end{align}
  
  % a)
  \begin{subex}
    Suppose we roll two 6-sided dice. Consider the events:
    \[
      D = \text{`odd on die 1'} \quad
      E = \text{`odd on die 2'} \quad
      F = \text{`odd sum'}
    \]
    Are $D$, $E$, and $F$ pairwise independent? Are they mutually independent?	
  \end{subex}

  % b)
  \begin{subex}
    Consider the Venn diagram in figure \ref{fig:venn}. $A$, $B$ and $C$ are
    the overlapping circles and the probabilities of each region are as marked.
    Does equation \ref{eq:pairwise-indep} hold? Are the events $A$, $B$, $C$
    mutually independent?
  \end{subex}
  
  % c)
  \begin{subex}
  For families with $n$ children, the events ``the family has children of both
  sexes'' and ``there is at most one girl'' are independent. What is $n$?
  \end{subex}
\end{exercise}

%-
\begin{figure}\centering
  \includegraphics[width=.2\textwidth]{03-venn}	
  \caption{Venn diagram for problem 4 (b)\label{fig:venn}}
\end{figure}
%-

\begin{exercise}[Trees of cards]
  There are $8$ cards in a hat:
  \[
    \{ 1\heartsuit, 1\spadesuit, 1\diamondsuit, 1\clubsuit, 2\heartsuit,
    2\spadesuit, 2\diamondsuit, 2\clubsuit\}.
  \]
  You draw one card at random. If its rank is 1 you draw one more card;
  if its rank is two you draw two more cards. Let $X$ be the sum of the
  ranks on all the cards you have drawn. Find $E(X)$.	
\end{exercise}

%---------------

\begin{exercise}[Seating arrangements]
  A total of $n$ people take their seats around a circular table with $n$
  chairs. No two people have the same height. What is the expected number of
  people who are shorter than both of their immediate neighbours?
\end{exercise}

%---------------

\begin{exercise}[Negative binomial distribution]
  Assume that a number of independent trials, each with a probability of
  success of $p$, $0 < p < 1$, are performed until $q$ successes are
  registered. Let $X$ be equal to the number of trials required, then
  \[
    P(X = n) 
      = {{n-1} \choose {q-1}} p^{q} (1 - p)^{n-q} 
      \qquad n = q, q+1, \dots
  \]
  Any RV $X$ whose probability distribution is given by the above is said to be
  a \emph{negative binomial RV} with parameters $(q,p)$. Compute the
  expectation and variance of this RV with parameters $(q,p)$.
\end{exercise}

\vfill\creditspracticequestions
\end{document}