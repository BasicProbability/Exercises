\documentclass[a4paper,10pt,landscape,twocolumn]{scrartcl}
\newcommand\problemset{5}
\newcommand\deadline{Wednesday October 5th, 22:00h}
\newif\ifcomments
\commentsfalse % hide comments
%\commentstrue % show comments
\usepackage[english]{{../exercises}}
\usepackage{graphicx, booktabs}
\DeclareMathOperator{\Cov}{Cov}
\DeclareMathOperator{\Cor}{Cor}
\DeclareMathOperator{\Var}{Var}
\begin{document}

\homeworkproblems

{\sffamily\noindent
  Your homework must be handed in \textbf{electronically via
  \href{\canvasURL}{Canvas} before \deadline}. This deadline is strict and late
  submissions are graded with a 0. At the end of the course, the lowest of your
  7 weekly homework grades will be dropped. You are strongly encouraged to work
  together on the exercises, including the homework. However, after this
  discussion phase, you have to write down and submit your own individual
  solution. Numbers alone are never sufficient, always motivate your answers.
}

%%%%%%%%%%%%%%%%%%%%%%%%%%%%%%%%%%%%%%%%%%%%%%%%%%%

\begin{exercise}[Markov's inequality (2pt)]
  Let $X$ be a random variable and $a > 0$ some real constant. Prove 
  \emph{Markov's Inequality}:
  \[
    P(|X| \ge a) \le \frac{E[|X|]}{a},
  \]
  where $|X|$ is the
  \href{https://en.wikipedia.org/wiki/Absolute\_value}{absolute\_value} of $X$.

\end{exercise}

%---------------

\begin{exercise}[Poisson distribution (3.5pt)]
  The \emph{Poisson distribution} models how often some event happens in a
  given period of time. Its probability mass function is given by
  \[
    P(X = k) 
      = {\frac {\lambda ^{k}e^{-\lambda }}{k!}}, 
      \qquad \lambda \in \mathbb{R}_{>0}, 
      \quad k=0,1,2,\dots
  \]
  where $\lambda$ is the distribution's only parameter. Let $X_1^N := X_1,
  \dots, X_N$ be $N$ independent and identically distributed (i.i.d.) random
  variables following a $\text{Poisson}(\lambda)$ distribution.
  
  % a)
  \begin{subex}[1pt]
    Let $T(x_1, \dots, x_N) = \sum_{i=1}^N x_i$ be a statistic. Show that $T$
    is a sufficient statistic for the Poisson distribution. \emph{Hint: use the
    Factorization Theorem~5.13 from the script).}
  \end{subex}
  
  % b)
  \begin{subex}[1pt]
    Find the log-likelihood $\mathcal L_x(\lambda) = \ln P(X_1^N = x_1^N\mid
    \lambda)$ of the data.
  \end{subex}
  
  % c)
  \begin{subex}[0.5pt]
    Find the derivative $\frac{\partial}{\partial \lambda} \mathcal L(\lambda)$
    of the log-likelihood.
  \end{subex}
  
  % d)
  \begin{subex}[1pt]
    Show that the maximum-likelihood estimate for $\lambda$ is
    \[
      \lambda_\textsc{ml} 
        = \text{argmax}_\lambda P(X_1^N = x_1^N \mid \lambda) 
        = \frac{1}{N}\sum_{i=1}^N x_i.
    \]
  \end{subex}	
\end{exercise}

%--
\begin{figure}
  \includegraphics[width=.5\textwidth]{05-distributions}
  \caption{Distributions of the three models using the
  $\textsc{ml}$-parameters. The grey bars in the background show the empirical
  distribution of the observations, i.e.\ the relative frequency of all
  observations in $\mathcal{D}$.
  %-----
  \label{fig:models}}
\end{figure}
%-

\begin{exercise}[Three models (4.5pt)]
  Once upon a time, three data enthusiasts, Alice, Bob and Charlie, were asked
  to look into a dataset that was as small as it was confidential. The client
  could not provide any information about the dataset, so just gave the data:
  \[
    \mathcal{D} = \{ 1, 1, 2, 4, 2, 1, 3, 2, 2 \}.
  \]
  Puzzled, all three came up with different explanations of the data. Alice
  thought the data was drawn from a geometric distribution, Bob proposed a
  binomial distribution and Charlie a Poisson distribution (see Problem~2).
  More precisely, they assume that the data is generated by 
  $N = |\mathcal D|=9$ i.i.d. random variables, but everyone proposes different
  ones:
  \begin{align*}
    \text{Model Alice:} 
      \quad A_i &\sim \text{Geom}(\pi), 
        \mbox{defined as $P(A_i=k):=\pi \cdot (1-\pi)^{k-1}$.}
    \\
	\text{Model Bob:} 
      \quad B_i &\sim \text{Binom}(n, \theta), 
      \quad n=4
    \\
	\text{Model Charlie:} 
      \quad C_i &\sim \text{Poisson}(\lambda),
  \end{align*}
  for $i = 1, \dots, N$. So Alice, Bob and Charlie suppose that the observation
  $x_1 \in \mathcal D$ is the value taken by the RV $A_1, B_1$ and $C_1$
  respectively.

  % a)	
  \begin{subex}[1pt]
    Bob has already decided on one of the parameters: $n=4$. But all remaining
    parameters have to be estimated from the data. Calculate the
    \textsc{ml}-estimates for $\pi$, $\theta$ and $\lambda$ from the data. The
    distributions of the resulting models are shown in Figure \ref{fig:models}.
    \emph{Hint: use this week's board questions, this week's (updated) practice
    problem 5.3 and homework exercise~5.2}.
  \end{subex}

  % b)	
  \begin{subex}[1pt]
    Calculate the log-likelihood of the data $\mathcal D$ for each of the three
    models, using the \textsc{ml}-estimates from (a) as parameters. Which model
    gives the best explanation of the data, i.e.\ assigns the data the highest
    likelihood? \emph{Hint: you might find Table \ref{table} useful}.
  \end{subex}

  % c)	
  \begin{subex}[1pt]
    Alice, Bob and Charlie wonder how their models differ. Show that all models
    have the same expectation when parametrised with their respective MLE
    parameters. Observe that this expectation is equal to the sample mean
    $\overline x := \frac{1}{N} \sum_{j=1}^N x_j$, where $x_j$ is an
    observation. However, the models have different variances: show that
    $\Var(B_i) < \Var(C_i) \leq \Var(A_i)$ if $\overline x \ge 2$. 	
  \end{subex}
  
  % d)
  \begin{subex}[0.5pt]
    In order to better decide which model is the best, the client provides some
    more data. The full data set now is:
    \[
      \mathcal D' = \mathcal D \cup \{ 0, 4, 5 \}
    \]
    Which model(s) cannot account for this data? In other words, under which
    model(s) does this data get likelihood 0?
  \end{subex}
	
  % e)
  \begin{subex}[1pt]
    Let's call the parameter of the one remaining model $\gamma_\textsc{ml}$.
    Its inventor actually had a competing hypothesis for the parameter:
    $\gamma_\textsc{ml}' = 1.25 \times \gamma_\textsc{ml}$, which she deemed
    slightly less likely: $P(\gamma_\textsc{ml}') = 0.4$ while
    $P(\gamma_\textsc{ml}) = 0.6$. Which of these two parameters should she
    prefer after observing the full data set $\mathcal{D'}$? Answer that
    question by calculating the posterior $P(\gamma \mid \mathcal{D'})$.
  \end{subex}
  
  \bigbreak\noindent
  \emph{Working with actual data means many computations. That's why you
  normally want to do this using a computer --- not the human, but the
  electronic one. You can learn more about that in the follow-up course
  \emph{Basic Probability: programming}.}
\end{exercise}

%-
\begin{table}
  \begin{tabular}{l c c c}
  \toprule
    & 	$X \sim \text{Geom}(\pi_\textsc{ml})$
    &	$X \sim \text{Binom}(4, \theta_\textsc{ml})$
    &	$X \sim \text{Poisson}(\lambda_\textsc{ml})$ \\\midrule
    $\ln P(X = 0)$ &$-\infty$ &$-2.7726$ &$-2.0000$ \\
    $\ln P(X = 1)$ &$-0.6931$ &$-1.3863$ &$-1.3069$ \\
    $\ln P(X = 2)$ &$-1.3863$ &$-0.9808$ &$-1.3069$ \\
    $\ln P(X = 3)$ &$-2.0794$ &$-1.3863$ &$-1.7123$ \\
    $\ln P(X = 4)$ &$-2.7726$ &$-2.7726$ &$-2.4055$ \\
    $\ln P(X = 5)$ &$-3.4657$ &$-\infty$ &$-3.3218$ \\\bottomrule
  \end{tabular}
  \caption{Some log-probabilities for the three models \label{table}}
\end{table}
%-

\end{document}