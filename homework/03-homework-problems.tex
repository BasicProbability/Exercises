\documentclass[a4paper,10pt,landscape,twocolumn]{scrartcl}
\newcommand\problemset{3}
\newcommand\deadline{Wednesday September 19th, 21:00h}
\newif\ifcomments
\commentsfalse % hide comments
%\commentstrue % show comments
\usepackage[english]{{../exercises}}
\begin{document}

\homeworkproblems

{\sffamily\noindent
  Your homework must be handed in \textbf{electronically via
  \href{\canvasURL}{Canvas} before \deadline}. This deadline is strict and late
  submissions are graded with a 0. At the end of the course, the lowest of your
  7 weekly homework grades will be dropped. You are strongly encouraged to work
  together on the exercises, including the homework. However, after this
  discussion phase, you have to write down and submit your own individual
  solution. Numbers alone are never sufficient, always motivate your answers.
}

%%%%%%%%%%%%%%%%%%%%%%%%%%%%%%%%%%%%%%%%%%%%%%%%%%%%%%%%%%%%

\begin{exercise}[Expectation and variance (1pt)]
  Suppose $X$ is a random variable with $E[X] = 5$ and $\mathrm{Var}[X] = 7$.	
  
  % a)
  \begin{subex}[0.5pt]
    Compute $E[(2+X)^2]$
  \end{subex}	
  
  % b)  
  \begin{subex}[0.5pt]
    Compute $\mathrm{Var}(4+3X)$.
  \end{subex}
\end{exercise}


%---------------

\begin{exercise}[A probability urn (2.5pt)]
  A jar contains $r$ red and $b$ blue balls. Balls are taken out randomly until
  a blue ball is first drawn. At that point the experiment stops. Each drawn
  ball is put back before picking a new one.
  
  % a)
  \begin{subex}[1pt]
    What is the probability that you will draw exactly $k$ balls? 
  \end{subex}
  
  % b)
  \begin{subex}[1.5pt]
    What is the probability that you will draw \emph{at least} $k$ balls? 
    \emph{Hint: if you can't simplify your expression directly, look up the
    \href{https://en.wikipedia.org/wiki/Geometric_series}{geometric series}.}
  \end{subex}
\end{exercise}

%---------------

\begin{exercise}[Another probability urn (2.5pt)]
  Assume that $k$ balls are randomly picked (without replacement) from an urn
  containing $N$ balls labelled from $1$ to $N$. Let $X$ be the largest label
  present in a draw. For example: if you have drawn balls $\{1, 15, 9, 3, 14\}$
  then $X$ takes on the value $15$. Find an expression for the cumulative
  distribution function.
\end{exercise}

%---------------

\begin{exercise}[Two coins and a die (4pt)]
  You have two (fair) coins and a (fair) 4-sided die. Let $X$ be the number of
  heads after flipping the two coins and let $Y$ be the result of rolling the
  die. Let $Z$ be the average of $X$ and $Y$.

  % a)
  \begin{subex}[1.5pt]
    Find the standard deviations of $X$, $Y$ and $Z$.	
  \end{subex}
  
  % b)
  \begin{subex}[1pt]
    Draw a graph of the probability mass function and the cumulative
    distribution function of $Z$.
  \end{subex}
  
  % c)
  \begin{subex}[1.5pt]
    You play the following game. If $2X \ge Y$, you win $X^2$ euros and
    otherwise you lose 1 euro. What is your expected total gain or loss after
    playing this game $40$ times?
  \end{subex}
\end{exercise}

\end{document}