\documentclass[a4paper,10pt,landscape,twocolumn]{scrartcl}
\newcommand\problemset{4}
\newcommand\deadline{Wednesday September 28th, 22:00h}
\newif\ifcomments
\commentsfalse % hide comments
%\commentstrue % show comments
\usepackage[english]{{../exercises}}
\usepackage{booktabs}
\DeclareMathOperator{\Cov}{Cov}
\DeclareMathOperator{\Cor}{Cor}
\DeclareMathOperator{\Var}{Var}
\begin{document}

\homeworkproblems

{\sffamily\noindent
  Your homework must be handed in \textbf{electronically via
  \href{\canvasURL}{Canvas} before \deadline}. This deadline is strict and late
  submissions are graded with a 0. At the end of the course, the lowest of your
  7 weekly homework grades will be dropped. You are strongly encouraged to work
  together on the exercises, including the homework. However, after this
  discussion phase, you have to write down and submit your own individual
  solution. Numbers alone are never sufficient, always motivate your answers.
}

%%%%%%%%%%%%%%%%%%%%%%%%%%%%%%%%%%%%%%%%%%%%%%%%%%%

\begin{exercise}[Spam filters (1pt)]
  Suppose that 20\% of the incoming emails are junk mail (in fact, it's more).
  A company is developing a spam filter that already manages to correctly label
  all junk mail as spam. But in doing so, it also wrongly labels some innocent
  emails as spam. Let's call the probability of wrongly labelling an innocent
  email as spam the \emph{inaccuracy} $\alpha$.
  
  The spam filter's inaccuracy can be decreased in such a way that all actual
  spam will still be intercepted. To what value should $\alpha$ at least be
  decreased if the company wants to be at least 99.9\% sure that an email
  labelled as spam is actually junk mail?
\end{exercise}

%---------------

\begin{exercise}[A joint probability vase (4pt)]
  Consider a vase with 3 red balls, 2 white balls and 5 blue balls. You draw 5
  balls uniformly at random without replacement and ignore their order. Let $X$
  be the number of red balls and $Y$ be the number of white balls.
  
  % a)
  \begin{subex}[1pt]
    Let $C(x,y)$ be	number of outcomes in which $X=x$ and $Y=y$. Fill in the
    following table with all values $C$ can take.	
	\[\small{
		\begin{tabular}{c c c c}
		\toprule
		$C(X,Y)$&$Y=0$	&$Y=1$ &$Y=2$\\\midrule
		$X=0$	&		&  		&\\
		$X=1$	&		&		&\\
		$X=2$	&		&		&\\
		$X=3$	&		&		&\\\bottomrule		
		\end{tabular}}
	\]
	Justify that $P_{X,Y}(x,y) = {10 \choose 5}^{-1} C(x,y)$.
  \end{subex}
  
  % b)
  \begin{subex}[1pt]
    Provide a table of the cdf $F_{XY}(x,y) = P((X,Y) \le (x,y))$. (You might
    want to keep things simple and use $C$ in some way.)
  \end{subex}
  
  % c)
  \begin{subex}[0.5pt]
    Let $A$ be the event that you have drawn 1, 2 or 3 red balls and at most 1
    white ball. Calculate $P_{XY}(A)$ from the table.
  \end{subex}
  
  % d)
  \begin{subex}[0.5pt]
    Calculate $P_{XY}(A)$ again, and indicate how you use $F_{XY}$	
  \end{subex}
  
  % e)  
  \begin{subex}[0.5pt]
    Find the marginal distributions $P_X$ and $P_Y$.	
  \end{subex}

  % f)  
  \begin{subex}[0.5pt]
    Are $X$ and $Y$ independent?	
  \end{subex}
\end{exercise}

%---------------

\begin{exercise}[3pt]
  Let $P_{XY}$ be the joint distribution of $X$ and $Y$ whose marginal
  distributions are $X \sim \text{Binomial}(2, 0.5)$ and 
  $Y \sim \text{Bernoulli}(0.8)$. The joint distribution satisfies 
  $P_{XY}(2,0) = 0$ and $P_{XY}(1,1) = c$ for $c\in \mathbb R$.

  % a)
  \begin{subex}[1pt]
    Find the joint distribution of $X$ and $Y$ and calculate $\Cov(X,Y)$.
  \end{subex}
  
  % b)
  \begin{subex}[1pt]
  Calculate $\Cor(X,Y)$.	
  \end{subex}
  
  % c)
  \begin{subex}[1pt]
    Can we choose $c$ in such a way that $X$ and $Y$ become independent? Give
    $c$ if it exists or otherwise show it cannot exist.
  \end{subex}
\end{exercise}

%---------------

\begin{exercise}[Deterministic random variables (2pt)]
  A (discrete) random variable $Y$ is called \emph{constant, degenerate} or
  \emph{deterministic} if there exists an outcome $a$ such that $P_Y(a) = 1$.

  % a)
  \begin{subex}[1pt]
    Show that $X$ is constant if and only if $\Var[X] = 0$.	
  \end{subex}
  
  % b)
  \begin{subex}[1pt]
    Let $Y$ be constant and let $X$ be not constant, and further suppose that
    $E[X]$ is finite. Show that $\Cov(X,Y) = 0$.
  \end{subex}
\end{exercise}

\end{document}