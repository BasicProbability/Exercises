\documentclass[10pt, a5paper]{scrartcl}
\newcommand\problemset{4}
\newif\ifcomments
\commentsfalse % hide comments
%\commentstrue % show comments
\usepackage[english]{../../exercises}
\usepackage{amsmath,amssymb}
\usepackage[margin=1.7cm]{geometry}
\DeclareMathOperator{\var}{Var}
\DeclareMathOperator{\cov}{Cov}
\DeclareMathOperator{\cor}{Cor}
\begin{document}
\boardquestions

\begin{exercise}[Variances]
  
  % a)
  \begin{subex}
    Prove that if $X \sim \text{Bernoulli}(p)$, then $\var(X)= p(1-p)$.
  \end{subex}
  
  % b)
  \begin{subex}
    Prove that if $X \sim \text{Bin}(n, p)$, then $\var(X)= np(1-p)$.
  \end{subex}
  
  % c)
  \begin{subex}
    Suppose $X_1, X_2, \ldots, X_n$ are independent and all have the same standard
    deviation $\sigma$. Let $\overline{X}$ be the average of 
    $X_1, X_2, \ldots, X_n$. What is the standard deviation of $\overline{X}$? What
    does this mean?
  \end{subex}
\end{exercise}

%---------------

\begin{exercise}[Covariance]
  \begin{subex}
    Flip a fair coin 3 times. Let $X$ be the number of heads in the first 2 flips
    and let $Y$ be the number of heads in the last 2 flips. Give a table describing
    the joint distribution of $X$ and $Y$ and directly compute $\cov(X,Y)$.
  \end{subex}
  
  \begin{subex}
    Let $X_1, X_2, X_3$ be the results of the three fair coin flips and let $X$ and $Y$
    as before. Compute $\cov(X,Y)$ without first using the joint distribution.
  \end{subex}

\end{exercise}

%---------------

\begin{exercise}[More covariance]
  Toss a fair coin $2n+t$ times. Let $X$ be number of heads in the first $n+t$
  flips and let $Y$ be number of heads in the last $n+t$ flips. Compute
  $\cov(X,Y)$ and $\cor(X,Y)$.
\end{exercise}

\vfill\creditsboardquestions
\end{document}