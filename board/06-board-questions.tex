\documentclass[10pt, a5paper]{scrartcl}
\newcommand\problemset{6}
\newif\ifcomments
\commentsfalse % hide comments
%\commentstrue % show comments
\usepackage[english]{../../exercises}
\usepackage{amsmath,amssymb,amsthm}
\usepackage[margin=1.7cm]{geometry}
\newtheorem{theorem}{Theorem}
\newcommand{\len}{\ell}
\begin{document}
\boardquestions

%---------------

\begin{exercise}[Geometric EM]
  You are given a mixture model with mixture components $ c_{1}$ and $c_{2} $ which are linked to geometric
  distributions with parameters 
  \begin{align}
    \theta^{(0)}_{c_{1}} = 0.2, 
    \qquad \text{and} \qquad
    \theta^{(0)}_{c_{2}} = 0.6.
  \end{align}
  Both geometric distributions are of the form $p(X = k) = \theta (1-\theta)^k$.
  Assume that the latent variables $Y$ are identically and independently distributed and that $ P(Y=c_{1}|\Theta=\theta^{(0)}) = 0.2$.
  You observe the data set
  \begin{align*}
  x_1^4 = (0, 2, 2, 3).
  \end{align*}
  
  
  \begin{subex}
  What is the (marginal) log-likelihood of this data set under the model? Feel free to use
  calculators. Also, it might be helpful to collect all relevant probabilities in a table.
  \end{subex}
  
  \begin{subex} 
  Find the most likely mixture component for each data point.
  \end{subex}
  
  \begin{subex}
    Perform one EM iteration.    
  \end{subex}
  
  \begin{subex}
    Compute the marginal log-likelihood of the data with the updated parameters. The new value
  should be higher than the one computed in the beginning.
    
  \end{subex}
\end{exercise}


\end{document}